\documentclass[t]{beamer} %% alles am TOP ausrichten, c=CENTER ist default.
%% \documentclass[pausesections]{beamer}
% \documentclass[handout]{beamer}

% \usepackage{pgfpages}
%% \pgfpagesuselayout{4 on 1}[a4paper, border shrink 5mm]
% \pgfpagesuselayout{2 on 1}[a4paper, border shrink 5mm]
%% \pgfpagesuselayout{4 on 1}[a4paper, landscape]

% \usepackage[german]{babel}
\usepackage[latin1]{inputenc}
% \usepackage{times}
\usepackage[T1]{fontenc}
\usepackage{graphicx}
\usepackage{amscd}


\setlength{\parskip}{2mm}

% % don't show title and author in the footer
% \beamertemplatefootempty
% % Navigationssymbole ausblenden:
% \setbeamertemplate{navigation symbols}{}
% % \setbeamertemplate{footline}{}
% %% ohne Navigation:
% \beamertemplatenavigationsymbolsempty  

% don't show structure in header
% \beamertemplateheadempty
% fix block title font for acroread
% \setbeamerfont*{block title}{}

\mode<presentation>
{
  % \usetheme{Madrid} %ohne navigation
  % \usetheme{Warsaw}  
  % \usetheme{Hannover} 
  \usetheme{PaloAlto} 
  \setbeamercovered{transparent}
}


\title{  Verteiltes Wissen } 
%\subtitle{Reasoning with knowledge}
\author[Schubert, Schmid]{Danny Schubert, Jan Fabian Schmid}
\date[]{\today} 
\institute[UHH] {Seminar im Modul MvS\\
  Fachbereich Informatik \\ Universit�t Hamburg }

% Dies wird lediglich in den PDF Informationskatalog einf�gt. 
\subject{Slides}

\pgfdeclareimage[height=1.5cm]{university-logo}{UHH-Logo-cut}
\logo{\pgfuseimage{university-logo}}


% Folgendes sollte gel�scht werden, wenn man nicht am Anfang jedes
% Unterabschnitts die Gliederung nochmal sehen m�chte.
%\AtBeginSection[]
%{
%  \begin{frame}<beamer>
%    \frametitle{Outline}
%    \small\tableofcontents[currentsection,currentsubsection]
%  \end{frame}
%}


% Falls Aufz�hlungen immer schrittweise gezeigt werden sollen, kann
% folgendes Kommando benutzt werden:
% \beamerdefaultoverlayspecification{<+->}



\begin{document}

%%%%%%%%%%%%%%%%%%%%%%%%%%%%%%%%%%%%%%%%%%%%%%%%%%%%%% 
\begin{frame}
  \titlepage
\end{frame}

\section{Einf�hrung}
\subsection{Motivation}

\begin{frame}
  \frametitle{ The muddy children puzzle (Das schmutzige Kinder R�tsel?) }
	\begin{itemize}
		\item ToDo
	\end{itemize}
\end{frame}


\subsection{Theoretische Grundlagen}

\begin{frame}
  \frametitle{ Epistemische Logik }
  \begin{itemize}
  	\item Logikoperatoren: K, E, C (Wissensoperator, "Jeder wei�"-Operator, Gemeinsames Wissen)
  \end{itemize}
\end{frame}


\subsection{Probleme und  Fragen}

\begin{frame}
  \frametitle{ Konsens-Problem asynchroner Systeme }
    \begin{theorem}
    	Es existiert kein Kommunikationsprotokoll f�r ein Nachrichtensystem ohne begrenzte Nachrichtenlaufzeit mit dem gemeinsames Wissen zweier Prozesse �ber einen Bin�rwert erreicht werden kann.
    \end{theorem}
  \begin{itemize}
	\item Eine schw�chere Dedfinition von gemeinsamen Wissen muss genutzt werden.
  \end{itemize}	
\end{frame}


\end{document}
