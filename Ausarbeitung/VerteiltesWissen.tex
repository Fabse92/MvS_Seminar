\documentclass{llncs}

\usepackage[ngerman]{babel}
\usepackage[utf8]{inputenc} % latin1 anstatt utf8 falls Umlaute defekt
\usepackage{graphicx}
\graphicspath{{img/}}
\usepackage{pdfsync}
\usepackage{float}
\usepackage{amssymb}
\usepackage{amsmath}
\usepackage{tikz}

\newtheorem{satz}{Satz}{}

\renewcommand{\labelitemi}{$\bullet$}

%\DeclareUnicodeCharacter{00A0}{ }

\begin{document}
\title{Verteiltes Wissen}
\subtitle{\small{ Seminar Modellierung verteilter Systeme (MvS)\\ Sommersemester 2016,}}
\author{Danny Schubert, Jan Fabian Schmid\\\email{\{5schuber,2schmid\}@informatik.uni-hamburg.de}}
\institute{Universität Hamburg}
\maketitle

\begin{abstract}
In einem verteilten System müssen die Prozesse sinnvolle Entscheidungen mit unvollständigen Wissen über das Gesamtsystem fällen.
Diese Seminar-Arbeit behandelt die Frage, wie Prozesse mit dem Wissen, das ihnen zur Verfügung steht, mittels logischer Schlussfolgerungen Informationen ableiten können.
Dazu wird eine formale Logik mit intuitiver Repräsentation vorgestellt.
Es zeigt sich, dass es eine große Rolle spielt, wie ein Prozess eine neue Information erhalten hat, also wie das Nachrichtensystem funktioniert. Das liegt daran, dass ein Prozess je nach Nachrichtensystem unterschiedliche Annahmen an das Wissen anderer Prozesse machen kann.
Das Problem, das durch asynchrone Nachrichtensysteme entsteht, kann auf das Konsens-Problem zurückgeführt werden.
Dieses Problem kann durch schwächere Forderungen an das Wissen in verteilten Systemen umgangen werden.
Am Ende der Arbeit wird noch deutlich gemacht, dass es nicht zu unterschätzen ist, wie viel einfacher Lösungen zu einem Problem gefunden werden können, wenn die Prozesse des Systems größere Freiheiten bei der Kommunikation miteinander haben.
\end{abstract}

\section{Einleitung}
Im Gegensatz zu einem einfachen zentralisierten System müssen die Prozesse in einem verteilten System Entscheidungen auf Grundlage unvollständiger Informationen treffen.
Die einzelnen Prozesse des verteilten Systems können nicht ohne Einschränkungen Kenntnis über die Zustände der anderen Prozesse und somit über den Systemzustand erlangen.
So muss der Informationsaustausch zweier Prozesse mit Hilfe von Nachrichten durchgeführt werden. Diese Kommunikation ist sehr aufwendig, zudem ist die Laufzeit der Nachrichten in vielen Systemen unbekannt.
Entscheidungen eines Prozesses müssen daher oft mit seinen unvollständigen, veralteten oder schlicht falschen lokalen Informationen über das Gesamtsystem getroffen werden.\\
In dieser Arbeit soll betrachtet werden wie lokale Prozesse mit Hilfe von Wissenslogik Informationen über das System ableiten können und unter welchen Umständen sie korrekte Entscheidungen treffen können.

\subsection{Motivation: Das \textit{cheating husbands}-Rätsel}
\label{motivation}
Um das Konzept von verteiltem Wissen zu veranschaulichen nutzen Moses et al. \cite{moses1986cheating} das \textit{cheating husbands}-Rätsel (vgl. \cite{moses1986cheating} S. 168 ff.).
Dabei handelt es sich um ein Induktions-Rätsel von dem es viele Varianten gibt (\textit{unfaithful wives problem}, \textit{blue eyes problem}, \textit{muddy children puzzle}). Wir werden dieses Beispiel als durchgehende Veranschaulichung der Theorie verwenden.\\
Das Rätsel wird im Rahmen einer Geschichte dargelegt: Die Königin Henrietta I von Atlantis möchte das Untreue-Problem in der Stadt Mamajorca lösen, unter dem die verheirateten Frauen leiden; die Ehemänner betrügen ihre Frauen mit anderen Frauen der Stadt.
Folgende Informationen können als gemeinsames Wissen unter der Bevölkerung der Stadt vorausgesetzt werden:
\begin{itemize}
	\item Die Königin sagt die Wahrheit.
	\item Alle Ehefrauen gehorchen der Königin.
	\item Die Ehefrauen sind perfekte Logiker.
	\item Ein Pistolenschuss kann in der ganzen Stadt gehört werden.
	\item Keine Ehefrau weiß, ob ihr Ehegatte sie betrügt.
	\item Jede Ehefrau weiß, welche der anderen Frauen betrogen worden sind.
\end{itemize}
Um das Untreue-Problem zu lösen, lässt Henrietta I alle Ehefrauen der Stadt auf dem Marktplatz zusammenkommen und verkündet dann vor ihnen, dass es mindestens einen untreuen Ehemann gibt ($n \ge 1$).
Die Königin verbietet (aufgrund der Etikette), dass die Ehefrauen miteinander über die untreuen Ehemänner reden.
Sollte eine Ehefrau jedoch herausfinden, dass ihr Gatte untreu war, so muss sie ihn um Mitternacht des selben Tages erschießen.\medskip

Bei diesem Verhaltensprotokoll der Königin handelt es sich um einen Lösungs-vorschlag für das Untreue-Problem.
Das Untreue-Problem wird korrekt gelöst, wenn irgendwann alle untreuen Ehemänner erschossen wurden, während alle treuen Ehemänner am Leben gelassen wurde.\\

Die Korrektheit des Verhaltensprotokolls wird im folgenden Satz ausgedrückt:\\
Wenn es $n$ untreue Ehemänner zum Zeitpunkt der Verkündung von Henrietta I gab, so geschieht zunächst $n-1$ Nächte lang nichts und dann werden alle untreuen Ehemänner in der n-ten Nacht erschossen.\medskip

Der Beweis des Satzes kann mittels vollständiger Induktion über die Anzahl der untreuen Ehemänner in der Stadt vollzogen werden:\\
\textbf{Induktionsanfang:} Angenommen es gäbe nur einen ($n=1$) untreuen Ehemann. Jede Ehefrau weiß, welche der anderen Ehefrauen betrogen wurden. Daher gäbe es eine Ehefrau, die betrogene Ehefrau, die wüsste, dass keine der anderen Ehefrauen betrogen wurden. Da es mindestens einen untreuen Ehemann gibt kann sie darauf schließen, dass dies ihr eigener Mann sein muss. Dementsprechend wird der eine untreue Ehemann noch am selben Tag also in der ersten Nacht erschossen.\\
\textbf{Induktionsvoraussetzung:} Die Behauptung gelte für den Fall mit $n$ untreuen Ehemännern.\\
\textbf{Induktionsschluss:} Gezeigt werden muss nun, dass die Behauptung auch für $n+1$ untreue Ehemänner gilt.
Jede betrogene Ehefrau weiß in diesem Fall von $n$ betrogenen Ehefrauen in der Stadt. Die nicht betrogenen Ehefrauen hingegen kennen alle $n+1$ betrogenen Ehefrauen.
Aufgrund der Induktionsvoraussetzung wissen die betrogenen Ehefrauen, dass alle ihr bekannten betrogenen Ehefrauen ihre Männer in der n-ten Nacht erschossen hätten, wenn es tatsächlich nur $n$ untreue Ehemänner geben würde.
Da dies nicht geschehen ist, wissen die betrogenen Ehefrauen, dass es $n+1$ untreue Ehemänner geben muss, und dass ihr eigener Ehemann somit einer davon ist.
Die betrogenen Ehefrauen mussten dementsprechend darauf warten, ob in der n-ten Nacht Schüsse zu hören sind, um zu entscheiden, ob sie ihren Ehemann erschießen müssen.
Nach der n-ten Nacht weiß jede betrogene Ehefrau, dass ihr Mann sie betrogen hat, sodass alle untreuen Ehemänner in der n+1-ten Nacht um Mitternacht erschossen werden. $\square$

Entscheidend bei diesem Szenario ist, dass die Königin die Information, dass es mindestens einen untreuen Ehemann gibt, zu gemeinsamen Wissen macht.
Ohne dieses Wissen gilt der Induktionsanfang nicht; tatsächlich werden die Ehefrauen nie herausfinden, ob ihr Ehemann untreu war, wenn diese Information nicht allgemein bekannt gemacht wird (vgl. \cite{kshemkalyani2011distributed} S.283).

\subsection{Probleme und Fragen}
Eines der zentralen Probleme beim Arbeiten mit verteiltem Wissen besteht in der Unmöglichkeit gemeinsames Wissen mehrerer Prozesse in einem asynchronen System herzustellen, wenn die Prozesse fehlerhaft handeln können oder wenn es keine obere Schranke für die Nachrichtenlaufzeit gibt (vgl.\cite{kshemkalyani2011distributed} S. 293 f.).
Um dennoch mit asynchronen Systemen in der Praxis arbeiten zu können, nutzt man daher Verfahren die eingeschränktes gemeinsames Wissen ermöglichen.\\
Weitere Fragestellungen des Themas beziehen sich auf die Art des Nachrichtensystems, welches sich auf das Wissen der lokalen Prozesse auswirkt und wie die Informationsbeziehungen zwischen den Prozessen visuell dargestellt werden können.

\subsection{Aufbau der Arbeit}
Im Kapitel \ref{Logik} werden zunächst die Logikoperatoren eingeführt, mit denen in verteilten Systemen Wissen abgeleitet werden kann.
Um die Beziehungen zwischen den Prozessen und die Entwicklung von Wissen in verteilten Systemen intuitiv darzustellen, werden in Kapitel \ref{Kripke-Modelle} Kripke-Modelle eingeführt. Kapitel \ref{sync_vs_async} geht der Frage nach, wie das ableitbare Wissen in einem System davon abhängt, ob das Nachrichtensystem synchron oder asynchron funktioniert. Um das Konsens-Problem in asynchronen Systemen zu vermeiden werden in Kapitel \ref{GemeinsamesWissen} schwächere Definitionen von gemeinsamen Wissen vorgestellt.
Abschließend wird die Arbeit in Kapitel \ref{Zusammenfassung} resümiert.
\section{Logik des Wissens}
\label{Logik}
Damit möglichst intuitiv Logik über das Wissen in verteilten Systemen betrieben werden kann, werden Logikoperatoren aus der epistemischen Logik genutzt, mit denen eine Logik (bspw. Aussagenlogik) erweitert werden kann.

Fagin et al. führen dazu zunächst das Konzept von \textbf{möglichen Welten} ein (vgl. \cite{fagin2003reasoning} S.15 ff.). 
Da ein einzelner Prozess in einem verteilten System nur eingeschränkte Informationen über die Zustände in den anderen Prozessen hat, kann es verschiedene Systemzustände geben, die er anhand seiner lokalen Informationen bzw. Zustände nicht voneinander unterscheiden kann. Die möglichen Welten $P_i$ werden daher definiert als: Die Vereinigung aller Systemzustände, die mit den vorhanden Informationen eines Prozesses $i$ vereinbar sind.
Mit Hilfe der möglichen Welten können Aussagen über eine Formel $\phi$ gemacht werden:
Ein Prozess $i$ \textit{weiß}, dass $\phi$ gilt ($K_i\phi$), wenn in allen Welten, die er für möglich hält, $\phi$ gilt.
%Die Syntax zur Wissenslogik kann entsprechend ausgedrückt werden als:
%\begin{align*}
%\phi &::= p | \neg\phi_1 | \phi_1 \land \phi_2 | K_i\phi
%\end{align*}
%Wobei $p\in P$ eine atomare Aussagenlogische Formel ist, $\phi_1$ und $\phi_2$ sind Formeln unserer Wissenslogik und $i \in \{1,...,n\}$ in einem verteilten System mit $n$ Prozessen.
Die Semantik zu den Logikoperatoren des Wissens kann mit Hilfe von \textbf{Kripke-Modellen} definiert werden (vgl. \cite{kshemkalyani2011distributed} S. 285 f.).
Ein Kripke-Modell $M$ zu einem verteilten System mit $n$ Prozessen und einer Menge atomarer Aussagen $\Phi$ ist ein Tupel $(S,\pi,\mathcal{K}_1,...,\mathcal{K}_n)$ mit der Menge möglicher Systemzustände $S$, der Belegungsfunktion $\pi$, die jeder atomaren Aussage in jedem Zustand $s$ einen Wahrheitswert zuordnet ($\forall s\in S, \pi(s):\Phi \rightarrow \{0,1\}$) und einer binären Relation $\mathcal{K}_i$ auf $S$ für jeden Prozess $i$, die jedem Prozess $i$ seine möglichen Welten, also die Menge von Systemzuständen die er nicht vom tatsächlichen Systemzustand unterscheiden kann, zuordnet.\medskip

Damit ergeben sich folgende Definitionen der Wissensoperatoren:
\begin{itemize}
\item (M,s) $\vDash \phi$, falls $\pi(s)(\phi) = true$
\item (M,s) $\vDash K_i(\phi)$, falls für alle t mit $(s,t) \in \mathcal{K}_i $ gilt: (M,t) $\vDash \phi$
\item (M,s) $\vDash E^1(\phi)$, falls (M,s) $\vDash\land_{i\in N}K_i(\phi)$
\item (M,s) $\vDash E^{k+1}(\phi)$ mit $k\ge 1$, falls (M,s) $\vDash\land_{i\in N}K_i(E^k(\phi))$
\item (M,s) $\vDash C(\phi)$, falls (M,s) $\vDash\land_{k\in Z^*}E^k(\phi)$
\end{itemize}

Dabei ist K der bereits erwähnte Wissensoperator, E der \textit{jeder weiß}-Operator und C der \textit{gemeinsames Wissen}-Operator.
Der \textit{jeder weiß}-Operator wird rekursiv für höhere Ebenen definiert: $E^1(\phi)$ bedeutet, dass jeder Prozess im System weiß, dass $\phi$ gilt; $E^2(\phi)$ bedeutet, dass jeder Prozess weiß, dass jeder Prozess weiß, dass $\phi$ gilt, und so weiter.
Der \textit{gemeinsames Wissen}-Operator kann intuitiv so verstanden werden, dass es definitiv für keinen Prozess einen Zweifel daran gibt, dass den anderen Prozessen $\phi$ bekannt ist. Dies gilt beispielsweise in unserer ersten Version des \textit{cheating husbands}-Rätsels für die Information, dass es mindestens einen untreuen Ehemann gibt, nachdem die Königin dies in der Gegenwart aller Ehefrauen verkündet hat. 
\section{Veranschaulichung mit Kripke-Modellen}
\label{Kripke-Modelle}
\subsection{Kripke-Modelle zum \textit{cheating husbands}-Rätsel}

%########################################################################################
% Chapter: Wissen in synchronen und asynchronen Systemen
%########################################################################################
\section{Wissen in synchronen und asynchronen Systemen}
\label{sync_vs_async}
%\begin{figure}[H]
%\centering
%      \includegraphics[width=0.8\textwidth]{def_async_sys.pdf}
%  \caption{ \cite{Panangaden1992} }
%\label{pic:fismahistorical}
%\end{figure}
Nachfolgend wird in diesem Kapitel beispielhaft erläutert, wie die Problematik des \textit{cheating husbands}-Rätsel für die Fälle der asynchronen und synchronen Übertragung behandelt werden.

%########################################################################################
% Section: Wissen in asynchronen Systemen
%########################################################################################
\subsection{Wissen in asynchronen Systemen}
\label{wissen_sync}
Für die asynchronen Systeme gibt es einen Zusatz in dem Rätsel der \textit{cheating husbands} (\cite{moses1986cheating} et al.). Es beginnt damit, dass Henrietta I verstirbt und damit ihre Tochter (Henrietta II) die Herrschaft von Mamajorca übernimmt. Der letzte Wunsch ihrer Mutter war es, dass Benachrichtungssystem für betrogene Ehefrauen fort zu führen. \\\\
Die Tochter kam der bitte ihrer Mutter nach. Jedoch vollführte sie eine Verbesserung des Systems. Um die Kommunikation zu erleichtern lies Henrietta II Briefkästen an jedem Haushalt in der ganzen Stadt anbringen. Nach dem errichten lies die Königin als erstes einen Brief an alle ausliefern in dem die Neuerungen erläutert wurden und die Eigenschaften des neuen Systems veranschaulicht wurden. Die erste Eigenschaft besagt, dass jeder Brief den die Königin verschickt, wird garantiert, zu einem \textit{nicht vorhersagbaren Zeitpunkt}, eine der Frauen erreichen. Somit ist die zweite Eigenschaft ableitbar aus der ersten. Sie besagt lediglich, dass keine Ankündigungen mehr auf dem Marktplatz gemacht werden müssten.\\\\
Der grundsätzliche Gedanke ist der gleiche. D.h. nach Erhalt der Nachricht wird der untreue Ehemann in der folgenden Nacht erschossen. Aufgrund des zeitversetzten Zustellens der Nachricht gibt es Zustände in einem System, welche nicht vorhersagbar sind. Jenes Problem wird nachfolgend in Theorem \ref{theo_async_1} erläutert.
\begin{theorem}[vgl. Theorem 2, \cite{moses1986cheating} S. 168]
\label{theo_async_1}
Wenn mehr wie ein untreuer Ehemann existiert und die originalen Anweisungen gelten,so sind  diese über einen asynchronen Kanal zu versenden, somit wird kein Ehemann erschossen.
\end{theorem}
\begin{proof}
Der Beweis ist durch Veranschaulichung des Ablaufes leicht zu erbringen. Durch die asynchrone Verteilung der Briefe und die Tatsache das es eine eventuelle Verteilung dieser Briefe gibt handelt es sich um "eventual common knowledge" (siehe Kap. \ref{GemeinsamesWissen}). Wird nun ein Brief von der Königin versendet, erhält jede Frau eventuell den Brief, der für sie bestimmt war und die besagt Frau weiß ebenfalls, dass es eventuell noch weitere Frauen gibt, die einen Brief erhalten haben könnten. Es gibt keine Sicherheit um für die Frauen um fest zu stellen, ob ihr Ehemann zweifelsfrei einer der untreuen ist. Zweifelsfrei kann sich dies deshalb nicht sagen lassen, da die Frau nicht weiß ob die ruhigen Nächte auf der Tatsache basieren, dass die anderen Briefe noch nicht zugestellt wurden oder ob es eine Reaktion andere Frauen auf den Erhalt eines Briefes ist.
\end{proof}
Aus dem Beweis resultiert, dass sich die Frauen auf Grund der Unsicherheit nie dazu entschließen würden, ihren Gemahlen zu erschießen, denn es könnte sich ja auch um einen Irrtum handeln, da noch nicht alle Briefe zugestellt wurden.
\begin{definition}
\begin{itemize}
\item (a,c) $\vDash \phi$ genau dann, wenn $\phi$ wahr ist in einem Schnitt c von der asynchronen Ausführung a.
\item (a,c) $\vDash K_i(\phi)$, genau dann, wenn 
$\forall(a',c'), ((a',c') \sim_i (a,c) \implies (a',c') \vDash \phi)$
\item (a,c) $\vDash E^0(\phi)$, genau dann, wenn (a,c) $\vDash \phi$
\item (a,c) $\vDash E^1(\phi)$, genau dann, wenn (a,c) $\vDash\land_{i\in N}K_i(\phi)$
\item (a,c) $\vDash E^{k+1}(\phi)$ mit $k\ge 1$, falls (a,c) $\vDash\land_{i\in N}K_i(E^k(\phi))$ mit $k\ge 1$
\item (a,c) $\vDash C(\phi)$, genau dann, wenn (a,c) $\vDash$ den größten Fixpunkt erfüllt mit $X=(E(X)\wedge \phi)$ gilt, $C(\phi) \land_{k\in Z^*}E^k(\phi)$
\end{itemize}
\end{definition}
Abbildung \ref{unsichereAbsprache} zeigt genau diesen Sachverhalt.

\begin{figure}
	\begin{tikzpicture}[thick,scale=0.2]
	\coordinate[label={180:\text{(1,0,0)}}] (UVL) at (0,0,12);
	\coordinate[label={0:\text{(1,1,0)}}] (UVR) at (12,0,12);
	\coordinate[label={180:\text{(0,0,0)}}] (OVL) at (0,12,12);
	\coordinate[label={-120:\text{(0,1,0)}}] (OVR) at (12,12,12);
	\coordinate[label={-86:\text{(1,0,1)}}] (UHL) at (0,0,0);
	\coordinate[label={0:\text{(1,1,1)}}] (UHR) at (12,0,0);
	\coordinate[label={180:\text{(0,0,1)}}] (OHL) at (0,12,0);
	\coordinate[label={0:\text{(0,1,1)}}] (OHR) at (12,12,0);
	
	\draw[fill=black] (UVL) circle (8pt);
	\draw[fill=black] (UVR) circle (8pt);
	\draw[fill=black] (OVL) circle (8pt);
	\draw[fill=black] (OVR) circle (8pt);
	\draw[fill=black] (UHL) circle (8pt);
	\draw[fill=black] (UHR) circle (8pt);
	\draw[fill=black] (OHL) circle (8pt);
	\filldraw[color=red] (OHR) circle (8pt);
	
	\draw (UVL) -- node[above] {2} (UVR);
	\draw (UVL) -- node[above left] {1} (OVL);
	\draw (UVL) -- node[left] {3} (UHL);
	\draw (OVR) -- node[above right] {2} (OVL);
	\draw (OVR) -- node[left] {3} (OHR);
	\draw (OVR) -- node[above left] {1} (UVR);
	\draw (OHL) -- node[left] {3} (OVL);
	\draw (OHL) -- node[above] {2} (OHR);
	\draw (OHL) -- node[left] {1} (UHL);
	\draw (UHR) -- node[left] {3} (UVR);
	\draw (UHR) -- node[left] {1} (OHR);
	\draw (UHR) -- node[above left] {2} (UHL);	
	
	\draw[line width=1.5pt,->] (22,8,12) -- node[above] {1. Nacht} (26,8,12); 	
	\end{tikzpicture}
	\qquad
	\hspace*{-0.7cm}
	\begin{tikzpicture}[thick,scale=0.2]
	\coordinate[label={180:\text{(1,0,0)}}] (UVL) at (0,0,12);
	\coordinate[label={0:\text{(1,1,0)}}] (UVR) at (12,0,12);
	\coordinate[label={180:\text{(0,0,0)}}] (OVL) at (0,12,12);
	\coordinate[label={-120:\text{(0,1,0)}}] (OVR) at (12,12,12);
	\coordinate[label={-86:\text{(1,0,1)}}] (UHL) at (0,0,0);
	\coordinate[label={0:\text{(1,1,1)}}] (UHR) at (12,0,0);
	\coordinate[label={180:\text{(0,0,1)}}] (OHL) at (0,12,0);
	\coordinate[label={0:\text{(0,1,1)}}] (OHR) at (12,12,0);
	
	\draw[fill=black] (UVL) circle (8pt);
	\draw[fill=black] (UVR) circle (8pt);
	\draw[fill=black] (OVL) circle (8pt);
	\draw[fill=black] (OVR) circle (8pt);
	\draw[fill=black] (UHL) circle (8pt);
	\draw[fill=black] (UHR) circle (8pt);
	\draw[fill=black] (OHL) circle (8pt);
	\filldraw[color=red] (OHR) circle (8pt);	
	
	\draw (UVL) -- node[above] {2} (UVR);
	\draw (UVL) -- node[left] {3} (UHL);
	\draw (OVR) -- node[left] {3} (OHR);
	\draw (OVR) -- node[above left] {1} (UVR);
	\draw (OHL) -- node[above] {2} (OHR);
	\draw (OHL) -- node[left] {1} (UHL);
	\draw[line width=1.5pt,->] (22,8,12) -- node[above] {2. Nacht} (26,8,12); 		
	\end{tikzpicture}
	\centering	
	\begin{tikzpicture}[thick,scale=0.2]
	\coordinate[label={180:\text{(1,0,0)}}] (UVL) at (0,0,12);
	\coordinate[label={0:\text{(1,1,0)}}] (UVR) at (12,0,12);
	\coordinate[label={180:\text{(0,0,0)}}] (OVL) at (0,12,12);
	\coordinate[label={-120:\text{(0,1,0)}}] (OVR) at (12,12,12);
	\coordinate[label={-86:\text{(1,0,1)}}] (UHL) at (0,0,0);
	\coordinate[label={0:\text{(1,1,1)}}] (UHR) at (12,0,0);
	\coordinate[label={180:\text{(0,0,1)}}] (OHL) at (0,12,0);
	\coordinate[label={0:\text{(0,1,1)}}] (OHR) at (12,12,0);
	
	\draw[fill=black] (UVL) circle (8pt);
	\draw[fill=black] (UVR) circle (8pt);
	\draw[fill=black] (OVL) circle (8pt);
	\draw[fill=black] (OVR) circle (8pt);
	\draw[fill=black] (UHL) circle (8pt);
	\draw[fill=black] (UHR) circle (8pt);
	\draw[fill=black] (OHL) circle (8pt);
	\filldraw[color=red] (OHR) circle (8pt);	
	
	\draw (UVL) -- node[above] {2} (UVR);
	\draw (UVL) -- node[left] {3} (UHL);
	\draw (OVR) -- node[left] {3} (OHR);
	\draw (OVR) -- node[above left] {1} (UVR);
	\draw (OHL) -- node[above] {2} (OHR);
	\draw (OHL) -- node[left] {1} (UHL);	
	\end{tikzpicture}
	\caption{Übergang vom ersten zum zweiten Tag im \textit{cheating husbands}-Rätsels mit ungenügender Kommunikation ausgehend vom Zustand (0,1,1).}
	\label{unsichereAbsprache}
\end{figure}
%########################################################################################
% Section: Wissen in synchronen Systemen
%########################################################################################

\subsection{Wissen in synchronen Systemen}
\label{wissen_async}




\begin{satz}[vgl. \cite{moses1986cheating} S. 168]
\label{pythagorean}
Wenn mehr wie ein untreuer Ehemann existiert und die originalen Anweisungen sind, diese über einen asynchronen Kanal zu versenden, dann wird kein Ehemann erschossen.
\end{satz}

\begin{figure}
\centering
\begin{tikzpicture}[thick,scale=0.2]
\coordinate[label={180:\text{(1,0,0)}}] (UVL) at (0,0,12);
\coordinate[label={0:\text{(1,1,0)}}] (UVR) at (12,0,12);
\coordinate[label={180:\text{(0,0,0)}}] (OVL) at (0,12,12);
\coordinate[label={-120:\text{(0,1,0)}}] (OVR) at (12,12,12);
\coordinate[label={-86:\text{(1,0,1)}}] (UHL) at (0,0,0);
\coordinate[label={0:\text{(1,1,1)}}] (UHR) at (12,0,0);
\coordinate[label={180:\text{(0,0,1)}}] (OHL) at (0,12,0);
\coordinate[label={0:\text{(0,1,1)}}] (OHR) at (12,12,0);

\draw[fill=black] (UVL) circle (8pt);
\draw[fill=black] (UVR) circle (8pt);
\draw[fill=black] (OVL) circle (8pt);
\draw[fill=black] (OVR) circle (8pt);
\draw[fill=black] (UHL) circle (8pt);
\draw[fill=black] (UHR) circle (8pt);
\draw[fill=black] (OHL) circle (8pt);
\filldraw[color=black] (OHR) circle (8pt);

\draw (UVL) -- node[above] {2} (UVR);
\draw (UVL) -- node[above left] {1} (OVL);
\draw (UVL) -- node[left] {3} (UHL);
\draw (OVR) -- node[above right] {2} (OVL);
\draw (OVR) -- node[left] {3} (OHR);
\draw (OVR) -- node[above left] {1} (UVR);
\draw (OHL) -- node[left] {3} (OVL);
\draw (OHL) -- node[above] {2} (OHR);
\draw (OHL) -- node[left] {1} (UHL);
\draw (UHR) -- node[left] {3} (UVR);
\draw (UHR) -- node[left] {1} (OHR);
\draw (UHR) -- node[above left] {2} (UHL);
\end{tikzpicture}
\caption{Ausgangssituation des \textit{cheating husbands}-Rätsels}
\label{ausgang}
\end{figure}

\section{Varianten des gemeinsamen Wissens}
\label{GemeinsamesWissen}

%########################################################################################
% Section: Epsilon common knowledge
%########################################################################################
\subsection{Epsilon common knowledge}
\label{epsilon_comm_know}

%########################################################################################
% Section: Wissen in asynchronen Systemen
%########################################################################################
\subsection{Eventual common knowledge}
\label{eventual_comm_know}

%########################################################################################
% Section: Wissen in asynchronen Systemen
%########################################################################################
\subsection{Timestamped common knowledge}
\label{timestamped_comm_know}

%########################################################################################
% Section: Wissen in asynchronen Systemen
%########################################################################################
\subsection{Concurrent common knowledge}
\label{concurrent_comm_know}

\section{Zusätzliche Kommunikation}
\label{Kommuniaktion}
In den bisherigen Betrachtungen der Varianten zum \textit{cheating-husbands}-Rätsel war der Informationsaustausch zwischen den Ehefrauen stark eingeschränkt.
Die Kommunikation einer Ehefrau war darauf beschränkt mitzuteilen, ob sie weiß, dass sie betrogen wurde.
Wenn sie dies mit Sicherheit sagen konnte, erschießt sie ihren Ehemann in der Nacht, die anderen Ehefrauen hörten den Schuss und wussten somit, dass eine andere Ehefrau sich sicher war, dass ihr Ehemann betrogen hat.\\
Wenn diese Beschränkung aufgehoben wird, können ohne die Mächtigkeit des Systems zu erhöhen, wesentlich effizientere Kommunikationsprotokolle genutzt werden, um das Problem zu lösen.
Um dies zu zeigen abstrahieren wir nun wieder von dem, als sehr entscheidend festgestellten, Nachrichtensystem. 
Nachrichten der Königin werden daher ohne Verzögerung verteilt, und jeder Ehefrau ist dies bekannt.
Bei der ursprünglichen Lösung aus der Motivation wurde das Problem in n Tagen Tagen gelöst, wobei n die Anzahl der untreuen Ehemänner ist.
Y. Moses et al. \cite{moses1986cheating} (vgl. S.175) zeigen, dass das Problem mit zusätzlicher Kommunikation mit einem Algorithmus gelöst werden kann, der höchstens drei Tage benötigt.
Die Mächtigkeit des Systems bleibt hier insofern gleich, als das weiterhin nur über Schüsse um Mitternacht kommuniziert werden kann. Dies entspricht einer binären Nachricht pro Nacht, entweder sind ein oder mehr Schüsse in der Nacht gefallen oder keiner.\\
Zunächst können wir mit einer an \cite{moses1986cheating} angelegten Überlegung beweisen, dass es kein Protokoll gibt, das weniger als drei Nächte unabhängig von der Probleminstanz benötigt:\\

Betrachten wir eine Probleminstanz mit k untreuen Ehemännern.
Im Allgemeinen kann es nun zwei verschiedene Anzahlen von untreuen Ehemännern geben, von denen die Ehefrauen wissen.
Entweder kennen sie alle k, wenn sie selbst nicht betrogen wurden, oder sie kennen $k-1$, wenn sie betrogen wurden.
Da keine der Ehefrauen weiß, ob sie betrogen wurde, wissen sie nur, dass die wahre Anzahl untreuer Ehemänner entweder $k_i$ oder $k_i+1$ ist.
Somit gibt es Ehefrauen die wissen, dass es $k-1$ oder $k$ und welche die wissen, dass es $k$ oder $k+1$ untreue Ehemänner gibt.
Wenn die wahre Anzahl untreuer Ehemänner bekannt ist, ist das System vollständig bestimmt, jede Ehefrau weiß dann, ob sie ihren Mann erschießen muss.
Jede Nacht kann allerdings nur eine binäre Information bekannt gemacht werden, daher kann im Allgemeinen jede Nacht nur einer der drei möglichen Anzahlen von untreuen Ehemännern ausgeschlossen werden.

So könnte ein Schuss in der ersten Nacht bedeuten, dass es entweder $x$ oder $y$ untreue Ehemänner sind. Eine zweite Nacht müsste dann noch abgewartet werden, um eindeutig zu bestimmen, welcher der beiden Fälle der Wahrheit entspricht. Wenn kein Schuss in der ersten Nacht fallen würde, wäre hier klar, dass es weder $x$ noch $y$ untreue Ehemänner sind. Es kann also durchaus Fälle geben, in denen das Problem in zwei Nächten gelöst ist. Es kann aber kein Protokoll geben, dass das Problem unabhängig von $k$ immer in weniger als drei Nächten löst.\\

\subsection{Protokoll mit zusätzlicher Kommunikation}

Die folgende leicht modifizierte Version des Kommunikationsprotokolls von Y. Moses et al. \cite{moses1986cheating} (vgl. S.175) ist eine optimale Lösung für das \textit{cheating husbands}-Rätsel mit zusätzlicher Kommunikation, in dem Sinne, dass es maximal drei Tage benötigt, um jegliche Probleminstanz korrekt zu lösen:

\begin{itemize}
	\item 1. Falls eine Ehefrau von $k_1$ (mit $k_1 \text{ mod } 3 = 1$) untreuen Ehemännern weiß, schießt sie in der ersten Nacht in die Luft.
	\item 2. Falls eine Ehefrau von $k_2$ (mit $k_2 \text{ mod } 3 = 2$) untreuen Ehemännern weiß und in der ersten Nacht kein Schuss gefallen ist, erschießt sie ihren Ehemann in der zweiten Nacht.
	\item 3. Falls eine Ehefrau von $k_0$ (mit $k_0 \text{ mod } 3 = 0$) untreuen Ehemännern weiß und in der ersten Nacht ein Schuss gefallen ist, erschießt sie ihren Ehemann in der zweiten Nacht.
	\item 4. Falls in den ersten beiden Nächten kein Schuss gefallen ist und $k_0 \ne0$, erschießen alle Ehefrauen ihren Mann in der dritten Nacht.
	\item 5. Falls in der ersten Nacht ein Schuss gefallen ist, in der zweiten Nach kein Schuss zu hören war, erschießen alle Ehefrauen, die in der ersten Nacht bereits geschossen haben, ihren Mann in der dritten Nacht. 
\end{itemize}
Das Protokoll kann folgendermaßen erklärt werden:\\
(Zu 1.) In der ersten Nacht wird die Fallunterscheidung gemacht, ob es Ehefrauen gibt, die von $k_1$ untreuen Ehemännern wissen oder nicht.
(Zu 2.) Falls dies nicht der Fall ist, ist bereits bekannt, dass es höchstens welche geben kann, die von $k_0$ oder $k_2$ untreuen Ehemännern wissen. Alle Ehefrauen, die von $k_2$ untreuen Ehemännern wissen, können ihren Mann nun töten, denn entweder wissen alle Ehefrauen von $k_2$ untreuen Ehemännern, dann sind alle betrogen wurden, oder die Ehefrauen, die nur $k_2$ untreue Ehemänner kennen, wissen von einem weniger, als die, die $k_0$ kennen.
(Zu 3.) Falls in der ersten Nacht geschossen wurde, können die Frauen, die von $k_0$ untreuen Ehemännern ihren Ehemann erschießen. Die Begründung verläuft analog zu der vorherigen.
(Zu 4.) Falls in den ersten beiden Nächten kein Schuss gefallen ist, wissen alle Ehefrauen von $k_0$ untreuen Ehemännern, sodass es entweder keinen einzigen untreuen Ehemann gibt, dann passiert nichts, oder alle Ehemänner waren untreu, dann werden alle erschossen.
(Zu 5.) Falls in der ersten Nacht Schüsse fielen in der zweiten jedoch nicht, kann es nur Ehefrauen geben, die von $k_1$ oder $k_2$ untreuen Ehemännern wissen. Aus einem analogen Argument, wie zu 1. und 2. können dann die Ehefrauen, die von $k_1$ untreuen Ehemännern wissen, ihren Ehemann erschießen.\\

Die Fähigkeit untereinander zu Kommunizieren ermöglicht es, dass in diesem Kommunikationsprotokoll keine Notwendigkeit darin besteht, dass die Königin verkündet, dass es mindestens einen untreuen Ehemann gibt. Tatsächlich funktioniert das Protokoll auch für den Fall, dass es keinen untreuen Ehemann gibt.
Mit Hilfe der Veranschaulichung von Kripke-Modellen können wir nun die korrekte Funktionsweise des Protokolls an dem Beispiel aus Kapitel \ref{Kripke-Modelle} betrachten:\\
Von der ersten zur zweiten Nacht wird der in Abbildung \ref{kom_(0,1,1)_1} dargestellte Übergang vollzogen, das Problem wird also direkt gelöst.
Der korrekte Zustand ist (0,1,1), somit ist die tatsächliche Anzahl der untreuen Ehemänner $k=2$. Die Ehefrauen 2 und 3 kennen $k_1 = 1$ untreue Ehemänner und Ehefrau 1 weiß von allen $k_2 = 2$ untreuen Ehemännern.
In der ersten Nacht schießen die beiden Ehefrauen mit $k_1$ in die Luft. Es gibt also mindestens einen untreuen Ehemann und höchstens zwei.
Dementsprechend sind in Abbildung \ref{kom_(0,1,1)_1} die Zustände (0,0,0) und (1,1,1) ausgeschlossen worden.

\begin{figure}
	\begin{tikzpicture}[thick,scale=0.2]
	\coordinate[label={180:\text{(1,0,0)}}] (UVL) at (0,0,12);
	\coordinate[label={0:\text{(1,1,0)}}] (UVR) at (12,0,12);
	\coordinate[label={180:\text{(0,0,0)}}] (OVL) at (0,12,12);
	\coordinate[label={-120:\text{(0,1,0)}}] (OVR) at (12,12,12);
	\coordinate[label={-86:\text{(1,0,1)}}] (UHL) at (0,0,0);
	\coordinate[label={0:\text{(1,1,1)}}] (UHR) at (12,0,0);
	\coordinate[label={180:\text{(0,0,1)}}] (OHL) at (0,12,0);
	\coordinate[label={0:\text{(0,1,1)}}] (OHR) at (12,12,0);
	
	\draw[fill=black] (UVL) circle (8pt);
	\draw[fill=black] (UVR) circle (8pt);
	\draw[fill=black] (OVL) circle (8pt);
	\draw[fill=black] (OVR) circle (8pt);
	\draw[fill=black] (UHL) circle (8pt);
	\draw[fill=black] (UHR) circle (8pt);
	\draw[fill=black] (OHL) circle (8pt);
	\filldraw[color=red] (OHR) circle (8pt);
	
	\draw (UVL) -- node[above] {2} (UVR);
	\draw (UVL) -- node[above left] {1} (OVL);
	\draw (UVL) -- node[left] {3} (UHL);
	\draw (OVR) -- node[above right] {2} (OVL);
	\draw (OVR) -- node[left] {3} (OHR);
	\draw (OVR) -- node[above left] {1} (UVR);
	\draw (OHL) -- node[left] {3} (OVL);
	\draw (OHL) -- node[above] {2} (OHR);
	\draw (OHL) -- node[left] {1} (UHL);
	\draw (UHR) -- node[left] {3} (UVR);
	\draw (UHR) -- node[left] {1} (OHR);
	\draw (UHR) -- node[above left] {2} (UHL);	
	
	\draw[line width=1.5pt,->] (22,8,12) -- node[above] {1. Nacht} (26,8,12); 	
	\end{tikzpicture}
	\qquad
	\hspace*{-0.7cm}
	\begin{tikzpicture}[thick,scale=0.2]
	\coordinate[label={180:\text{(1,0,0)}}] (UVL) at (0,0,12);
	\coordinate[label={0:\text{(1,1,0)}}] (UVR) at (12,0,12);
	\coordinate[label={180:\text{(0,0,0)}}] (OVL) at (0,12,12);
	\coordinate[label={-120:\text{(0,1,0)}}] (OVR) at (12,12,12);
	\coordinate[label={-86:\text{(1,0,1)}}] (UHL) at (0,0,0);
	\coordinate[label={0:\text{(1,1,1)}}] (UHR) at (12,0,0);
	\coordinate[label={180:\text{(0,0,1)}}] (OHL) at (0,12,0);
	\coordinate[label={0:\text{(0,1,1)}}] (OHR) at (12,12,0);
	
	\draw[fill=black] (UVL) circle (8pt);
	\draw[fill=black] (UVR) circle (8pt);
	\draw[fill=black] (OVL) circle (8pt);
	\draw[fill=black] (OVR) circle (8pt);
	\draw[fill=black] (UHL) circle (8pt);
	\draw[fill=black] (UHR) circle (8pt);
	\draw[fill=black] (OHL) circle (8pt);
	\filldraw[color=red] (OHR) circle (8pt);	
	
	\draw (UVL) -- node[above] {2} (UVR);
	\draw (UVL) -- node[left] {3} (UHL);
	\draw (OVR) -- node[left] {3} (OHR);
	\draw (OVR) -- node[above left] {1} (UVR);
	\draw (OHL) -- node[above] {2} (OHR);
	\draw (OHL) -- node[left] {1} (UHL);	
	\end{tikzpicture}
	\caption{Übergang vom ersten zum zweiten Tag im \textit{cheating husbands}-Rätsels mit zusätzlicher Kommunikation ausgehend vom Zustand (0,1,1).}
	\label{kom_(0,1,1)_1}
\end{figure}

Da es keine Ehefrau gibt, die von $k_0$ untreuen Ehemännern weiß, fällt in der zweiten Nacht kein Schuss.
Somit können auch alle Zustände ausgeschlossen werden, in denen es nur einen untreuen Ehemann gibt, sodass das Problem, wie in Abbildung \ref{kom_(0,1,1)_2} zu sehen, nach drei Nächten gelöst ist.

\begin{figure}
	\begin{tikzpicture}[thick,scale=0.2]
	\coordinate[label={180:\text{(1,0,0)}}] (UVL) at (0,0,12);
	\coordinate[label={0:\text{(1,1,0)}}] (UVR) at (12,0,12);
	\coordinate[label={180:\text{(0,0,0)}}] (OVL) at (0,12,12);
	\coordinate[label={-120:\text{(0,1,0)}}] (OVR) at (12,12,12);
	\coordinate[label={-86:\text{(1,0,1)}}] (UHL) at (0,0,0);
	\coordinate[label={0:\text{(1,1,1)}}] (UHR) at (12,0,0);
	\coordinate[label={180:\text{(0,0,1)}}] (OHL) at (0,12,0);
	\coordinate[label={0:\text{(0,1,1)}}] (OHR) at (12,12,0);
	
	\draw[fill=black] (UVL) circle (8pt);
	\draw[fill=black] (UVR) circle (8pt);
	\draw[fill=black] (OVL) circle (8pt);
	\draw[fill=black] (OVR) circle (8pt);
	\draw[fill=black] (UHL) circle (8pt);
	\draw[fill=black] (UHR) circle (8pt);
	\draw[fill=black] (OHL) circle (8pt);
	\filldraw[color=red] (OHR) circle (8pt);
	
	\draw (UVL) -- node[above] {2} (UVR);
	\draw (UVL) -- node[left] {3} (UHL);
	\draw (OVR) -- node[left] {3} (OHR);
	\draw (OVR) -- node[above left] {1} (UVR);
	\draw (OHL) -- node[above] {2} (OHR);
	\draw (OHL) -- node[left] {1} (UHL);	
	
	\draw[line width=1.5pt,->] (22,8,12) -- node[above] {2. Nacht} (26,8,12); 	
	\end{tikzpicture}
	\qquad
	\hspace*{-0.7cm}
	\begin{tikzpicture}[thick,scale=0.2]
	\coordinate[label={180:\text{(1,0,0)}}] (UVL) at (0,0,12);
	\coordinate[label={0:\text{(1,1,0)}}] (UVR) at (12,0,12);
	\coordinate[label={180:\text{(0,0,0)}}] (OVL) at (0,12,12);
	\coordinate[label={-120:\text{(0,1,0)}}] (OVR) at (12,12,12);
	\coordinate[label={-86:\text{(1,0,1)}}] (UHL) at (0,0,0);
	\coordinate[label={0:\text{(1,1,1)}}] (UHR) at (12,0,0);
	\coordinate[label={180:\text{(0,0,1)}}] (OHL) at (0,12,0);
	\coordinate[label={0:\text{(0,1,1)}}] (OHR) at (12,12,0);
	
	\draw[fill=black] (UVL) circle (8pt);
	\draw[fill=black] (UVR) circle (8pt);
	\draw[fill=black] (OVL) circle (8pt);
	\draw[fill=black] (OVR) circle (8pt);
	\draw[fill=black] (UHL) circle (8pt);
	\draw[fill=black] (UHR) circle (8pt);
	\draw[fill=black] (OHL) circle (8pt);
	\filldraw[color=red] (OHR) circle (8pt);		
	\end{tikzpicture}
	\caption{Übergang vom zweiten zum ersten Tag im \textit{cheating husbands}-Rätsels mit zusätzlicher Kommunikation ausgehend vom Zustand (0,1,1).}
	\label{kom_(0,1,1)_2}
\end{figure}

Betrachten wir den Spezialfall, dass (0,0,0) der wahre Zustand ist, mit Hilfe der Kripke-Modelle. Alle Ehefrauen kennen also $k_0 = 0$ untreue Ehemänner.
In der ersten Nacht sind keine Schüsse zu hören, sodass alle Zustände mit einem untreuen Ehemann ausgeschlossen werden können. Es ergibt sich die Situation aus \ref{kom_(0,0,0)_1}.

\begin{figure}
	\begin{tikzpicture}[thick,scale=0.2]
	\coordinate[label={180:\text{(1,0,0)}}] (UVL) at (0,0,12);
	\coordinate[label={0:\text{(1,1,0)}}] (UVR) at (12,0,12);
	\coordinate[label={180:\text{(0,0,0)}}] (OVL) at (0,12,12);
	\coordinate[label={-120:\text{(0,1,0)}}] (OVR) at (12,12,12);
	\coordinate[label={-86:\text{(1,0,1)}}] (UHL) at (0,0,0);
	\coordinate[label={0:\text{(1,1,1)}}] (UHR) at (12,0,0);
	\coordinate[label={180:\text{(0,0,1)}}] (OHL) at (0,12,0);
	\coordinate[label={0:\text{(0,1,1)}}] (OHR) at (12,12,0);
	
	\draw[fill=black] (UVL) circle (8pt);
	\draw[fill=black] (UVR) circle (8pt);
	\filldraw[color=red] (OVL) circle (8pt);
	\draw[fill=black] (OVR) circle (8pt);
	\draw[fill=black] (UHL) circle (8pt);
	\draw[fill=black] (UHR) circle (8pt);
	\draw[fill=black] (OHL) circle (8pt);
	\filldraw[color=black] (OHR) circle (8pt);
	
	\draw (UVL) -- node[above] {2} (UVR);
	\draw (UVL) -- node[above left] {1} (OVL);
	\draw (UVL) -- node[left] {3} (UHL);
	\draw (OVR) -- node[above right] {2} (OVL);
	\draw (OVR) -- node[left] {3} (OHR);
	\draw (OVR) -- node[above left] {1} (UVR);
	\draw (OHL) -- node[left] {3} (OVL);
	\draw (OHL) -- node[above] {2} (OHR);
	\draw (OHL) -- node[left] {1} (UHL);
	\draw (UHR) -- node[left] {3} (UVR);
	\draw (UHR) -- node[left] {1} (OHR);
	\draw (UHR) -- node[above left] {2} (UHL);	
	
	\draw[line width=1.5pt,->] (22,8,12) -- node[above] {1. Nacht} (26,8,12); 	
	\end{tikzpicture}
	\qquad
	\hspace*{-0.7cm}
	\begin{tikzpicture}[thick,scale=0.2]
	\coordinate[label={180:\text{(1,0,0)}}] (UVL) at (0,0,12);
	\coordinate[label={0:\text{(1,1,0)}}] (UVR) at (12,0,12);
	\coordinate[label={180:\text{(0,0,0)}}] (OVL) at (0,12,12);
	\coordinate[label={-120:\text{(0,1,0)}}] (OVR) at (12,12,12);
	\coordinate[label={-86:\text{(1,0,1)}}] (UHL) at (0,0,0);
	\coordinate[label={0:\text{(1,1,1)}}] (UHR) at (12,0,0);
	\coordinate[label={180:\text{(0,0,1)}}] (OHL) at (0,12,0);
	\coordinate[label={0:\text{(0,1,1)}}] (OHR) at (12,12,0);
	
	\draw[fill=black] (UVL) circle (8pt);
	\draw[fill=black] (UVR) circle (8pt);
	\filldraw[color=red] (OVL) circle (8pt);
	\draw[fill=black] (OVR) circle (8pt);
	\draw[fill=black] (UHL) circle (8pt);
	\draw[fill=black] (UHR) circle (8pt);
	\draw[fill=black] (OHL) circle (8pt);
	\filldraw[color=black] (OHR) circle (8pt);
	
	\draw (UHR) -- node[left] {3} (UVR);
	\draw (UHR) -- node[left] {1} (OHR);
	\draw (UHR) -- node[above left] {2} (UHL);	
	\end{tikzpicture}
	\caption{Übergang vom ersten zum zweiten Tag im \textit{cheating husbands}-Rätsels mit zusätzlicher Kommunikation ausgehend vom Zustand (0,0,0).}
	\label{kom_(0,0,0)_1}
\end{figure}

Zu diesem Zeitpunkt ist bereits jeder Ehefrau bewusst, dass (0,0,0) der korrekte Zustand ist. Wenn das Protokoll dennoch weiter ausgeführt wird ergibt sich der Übergang aus \ref{kom_(0,0,0)_2}, da keine Ehefrau von $k_2$ untreuen Ehemännern weiß.
Da in den ersten beiden Nächten kein Schuss gefallen ist, aber $k_0 = 0$ wird niemand erschossen, und das Problem wurde korrekt gelöst.

\begin{figure}
	\begin{tikzpicture}[thick,scale=0.2]
	\coordinate[label={180:\text{(1,0,0)}}] (UVL) at (0,0,12);
	\coordinate[label={0:\text{(1,1,0)}}] (UVR) at (12,0,12);
	\coordinate[label={180:\text{(0,0,0)}}] (OVL) at (0,12,12);
	\coordinate[label={-120:\text{(0,1,0)}}] (OVR) at (12,12,12);
	\coordinate[label={-86:\text{(1,0,1)}}] (UHL) at (0,0,0);
	\coordinate[label={0:\text{(1,1,1)}}] (UHR) at (12,0,0);
	\coordinate[label={180:\text{(0,0,1)}}] (OHL) at (0,12,0);
	\coordinate[label={0:\text{(0,1,1)}}] (OHR) at (12,12,0);
	
	\draw[fill=black] (UVL) circle (8pt);
	\draw[fill=black] (UVR) circle (8pt);
	\filldraw[color=red] (OVL) circle (8pt);
	\draw[fill=black] (OVR) circle (8pt);
	\draw[fill=black] (UHL) circle (8pt);
	\draw[fill=black] (UHR) circle (8pt);
	\draw[fill=black] (OHL) circle (8pt);
	\filldraw[color=black] (OHR) circle (8pt);
	
	\draw (UHR) -- node[left] {3} (UVR);
	\draw (UHR) -- node[left] {1} (OHR);
	\draw (UHR) -- node[above left] {2} (UHL);	
	
	\draw[line width=1.5pt,->] (22,8,12) -- node[above] {2. Nacht} (26,8,12); 	
	\end{tikzpicture}
	\qquad
	\hspace*{-0.7cm}
	\begin{tikzpicture}[thick,scale=0.2]
	\coordinate[label={180:\text{(1,0,0)}}] (UVL) at (0,0,12);
	\coordinate[label={0:\text{(1,1,0)}}] (UVR) at (12,0,12);
	\coordinate[label={180:\text{(0,0,0)}}] (OVL) at (0,12,12);
	\coordinate[label={-120:\text{(0,1,0)}}] (OVR) at (12,12,12);
	\coordinate[label={-86:\text{(1,0,1)}}] (UHL) at (0,0,0);
	\coordinate[label={0:\text{(1,1,1)}}] (UHR) at (12,0,0);
	\coordinate[label={180:\text{(0,0,1)}}] (OHL) at (0,12,0);
	\coordinate[label={0:\text{(0,1,1)}}] (OHR) at (12,12,0);
	
	\draw[fill=black] (UVL) circle (8pt);
	\draw[fill=black] (UVR) circle (8pt);
	\filldraw[color=red] (OVL) circle (8pt);
	\draw[fill=black] (OVR) circle (8pt);
	\draw[fill=black] (UHL) circle (8pt);
	\draw[fill=black] (UHR) circle (8pt);
	\draw[fill=black] (OHL) circle (8pt);
	\filldraw[color=black] (OHR) circle (8pt);
	
	\end{tikzpicture}
	\caption{Übergang vom ersten zum zweiten Tag im \textit{cheating husbands}-Rätsels mit zusätzlicher Kommunikation ausgehend vom Zustand (0,0,0).}
	\label{kom_(0,0,0)_2}
\end{figure}

%\begin{itemize}
%	\item 1. Falls eine Ehefrau von $k_0$ (mit $k_0 \text{ mod } 3 = 0$) untreuen Ehemännern weiß, schießt sie in der ersten Nacht in die Luft.
%	\item 2. Falls eine Ehefrau von $k_1$ (mit $k_1 \text{ mod } 3 = 1$) untreuen Ehemännern weiß und in der ersten Nacht kein Schuss gefallen ist, erschießt sie ihren Ehemann in der zweiten Nacht.
%	\item 3. Falls eine Ehefrau von $k_2$ (mit $k_2 \text{ mod } 3 = 2$) untreuen Ehemännern weiß und in der ersten Nacht ein Schuss gefallen ist, erschießt sie ihren Ehemann in der zweiten Nacht.
%	\item 4. Falls in den ersten beiden Nächten kein Schuss gefallen ist, erschießen alle Ehefrauen ihren Mann in der dritten Nacht.
%	\item 5. Falls in der ersten Nacht ein Schuss gefallen ist, in der zweiten Nach kein Schuss zu hören war (und $k_0 \ne0$ $<-$ \textbf{das stimmt nicht}), erschießen alle Ehefrauen, die in der ersten Nacht bereits geschossen haben, ihren Mann in der dritten Nacht. 
%\end{itemize}
%Das Protokoll kann folgendermaßen erklärt werden:\\
%Zu 2.: Hier wurde $k_0$ ausgeschlossen. Es gibt zwei Fälle, in denen niemand von $k_0$ einige hingegen von $k_1$ untreuen Ehemännern wissen: Manche Ehefrauen kennen $k_1$ und andere $k_2$ untreue Ehemänner oder alle Ehefrauen wissen von $k_1$ betrogenen Ehefrauen. Aufgrund der Modulo-Eigenschaft kann entweder $k_1 = k_0-2$ und $k_2 = k_0-1$ oder $k_1 = k_0+1$ und $k_2 = k_0+2$ sein. Entweder sind also alle Frauen betrogen worden, oder es gibt welche, die von einem untreuen Ehemann mehr als $k_1$ wissen. Die Ehefrauen, für die der zweite Fall des Protokolls zutrifft, müssen also betrogen worden sein.\\
%Zu 3.: Hier wurde $k_1$ ausgeschlossen,  gleiche Argumentation wie zu 2.\\
%Zu 4.: Sowohl $k_0$, als auch $k_1$ sind ausgeschlossen worden, jede Ehefrau weiß von $k_2$ untreuen Ehemännern, also sind alle Ehefrauen betrogen worden.\\
%Zu 5.: Gleiche Argumentation, wie bei 2. und 3., nur dass hier $k_2$ ausgeschlossen wurde. \\
%Es sind somit alle möglichen Fälle abgedeckt: Die Ehefrauen wissen von $k_i-1$ und $k_i$, von $k_i$ und $k_i+1$, nur von $k-1$, nur von $k$ oder nur von $k+1$ untreuen Ehemännern. Alle Regeln sind korrekt und somit auch das gesamte Kommunikationsprotokoll.\\
%
%Die Fähigkeit untereinander zu Kommunizieren ermöglicht es, dass in diesem Kommunikationsprotokoll keine Notwendigkeit darin besteht, dass die Königin verkündet, dass es mindestens einen untreuen Ehemann gibt. (Tatsächlich funktioniert das Protokoll auch für den Fall, dass es keinen untreuen Ehemann gibt. $<-$ \textbf{Das stimmt nicht!})
%Mit Hilfe der Veranschaulichung von Kripke-Modellen können wir nun die korrekte Funktionsweise des Protokolls an dem Beispiel aus Kapitel \ref{Kripke-Modelle} betrachten:\\
%Von der ersten zur zweiten Nacht wird der in Abbildung \ref{kom_(0,1,1)} dargestellte Übergang vollzogen, dass Problem wird also direkt gelöst.
%Der korrekte Zustand ist (0,1,1), somit ist die tatsächliche Anzahl der untreuen Ehemänner $k=2$. Die Ehefrauen 2 und 3 kennen $k_1 = 1$ untreue Ehemänner und Ehefrau 1 kennt $k_2 = 2$ untreue Ehemänner. Es gibt also keine Ehefrau, die $k_0$ untreue Ehemänner kennt, sodass in der ersten Nacht kein Schuss gefallen ist. Dementsprechend sind in Abbildung \ref{kom_(0,1,1)} die Zustände (0,0,0) und (1,1,1) ausgeschlossen worden.
%Die Zustände (1,0,0), (0,1,0) und (0,0,1) sind ebenso weggefallen, da es in ihnen eine Frau hätten geben müssen, die von keinem (also von $k_0$) untreuen Ehemann weiß. Alle Ehefrauen wissen also nach einer Nacht, dass sie sich in Zustand (0,1,1) befinden.
%
%\begin{figure}
%	\begin{tikzpicture}[thick,scale=0.2]
%	\coordinate[label={180:\text{(1,0,0)}}] (UVL) at (0,0,12);
%	\coordinate[label={0:\text{(1,1,0)}}] (UVR) at (12,0,12);
%	\coordinate[label={180:\text{(0,0,0)}}] (OVL) at (0,12,12);
%	\coordinate[label={-120:\text{(0,1,0)}}] (OVR) at (12,12,12);
%	\coordinate[label={-86:\text{(1,0,1)}}] (UHL) at (0,0,0);
%	\coordinate[label={0:\text{(1,1,1)}}] (UHR) at (12,0,0);
%	\coordinate[label={180:\text{(0,0,1)}}] (OHL) at (0,12,0);
%	\coordinate[label={0:\text{(0,1,1)}}] (OHR) at (12,12,0);
%	
%	\draw[fill=black] (UVL) circle (8pt);
%	\draw[fill=black] (UVR) circle (8pt);
%	\draw[fill=black] (OVL) circle (8pt);
%	\draw[fill=black] (OVR) circle (8pt);
%	\draw[fill=black] (UHL) circle (8pt);
%	\draw[fill=black] (UHR) circle (8pt);
%	\draw[fill=black] (OHL) circle (8pt);
%	\filldraw[color=red] (OHR) circle (8pt);
%	
%	\draw (UVL) -- node[above] {2} (UVR);
%	\draw (UVL) -- node[above left] {1} (OVL);
%	\draw (UVL) -- node[left] {3} (UHL);
%	\draw (OVR) -- node[above right] {2} (OVL);
%	\draw (OVR) -- node[left] {3} (OHR);
%	\draw (OVR) -- node[above left] {1} (UVR);
%	\draw (OHL) -- node[left] {3} (OVL);
%	\draw (OHL) -- node[above] {2} (OHR);
%	\draw (OHL) -- node[left] {1} (UHL);
%	\draw (UHR) -- node[left] {3} (UVR);
%	\draw (UHR) -- node[left] {1} (OHR);
%	\draw (UHR) -- node[above left] {2} (UHL);	
%	
%	\draw[line width=1.5pt,->] (22,8,12) -- node[above] {1. Nacht} (26,8,12); 	
%	\end{tikzpicture}
%	\qquad
%	\hspace*{-0.7cm}
%	\begin{tikzpicture}[thick,scale=0.2]
%	\coordinate[label={180:\text{(1,0,0)}}] (UVL) at (0,0,12);
%	\coordinate[label={0:\text{(1,1,0)}}] (UVR) at (12,0,12);
%	\coordinate[label={180:\text{(0,0,0)}}] (OVL) at (0,12,12);
%	\coordinate[label={-120:\text{(0,1,0)}}] (OVR) at (12,12,12);
%	\coordinate[label={-86:\text{(1,0,1)}}] (UHL) at (0,0,0);
%	\coordinate[label={0:\text{(1,1,1)}}] (UHR) at (12,0,0);
%	\coordinate[label={180:\text{(0,0,1)}}] (OHL) at (0,12,0);
%	\coordinate[label={0:\text{(0,1,1)}}] (OHR) at (12,12,0);
%	
%	\draw[fill=black] (UVL) circle (8pt);
%	\draw[fill=black] (UVR) circle (8pt);
%	\draw[fill=black] (OVL) circle (8pt);
%	\draw[fill=black] (OVR) circle (8pt);
%	\draw[fill=black] (UHL) circle (8pt);
%	\draw[fill=black] (UHR) circle (8pt);
%	\draw[fill=black] (OHL) circle (8pt);
%	\filldraw[color=red] (OHR) circle (8pt);		
%	\end{tikzpicture}
%	\caption{Übergang vom ersten zum zweiten Tag im \textit{cheating husbands}-Rätsels mit zusätzlicher Kommunikation ausgehend vom Zustand (0,1,1).}
%	\label{kom_(0,1,1)}
%\end{figure}
%
%Betrachten wir den Spezialfall, dass (0,0,0) der wahre Zustand ist, mit Hilfe der Kripke-Modelle. Alle Ehefrauen kennen also $k_0 = 0$ untreue Ehemänner.
%In der ersten Nacht sind also Schüsse zu hören, da niemand drei untreue Ehemänner kennen kann, wird Zustand (1,1,1) ausgeschlossen, die Zustände mit zwei untreuen Ehemännern können dann auch ausgeschlossen werden, da es entweder keinen oder einen untreuen Ehemann geben muss.
%
%
%\begin{figure}
%	\begin{tikzpicture}[thick,scale=0.2]
%	\coordinate[label={180:\text{(1,0,0)}}] (UVL) at (0,0,12);
%	\coordinate[label={0:\text{(1,1,0)}}] (UVR) at (12,0,12);
%	\coordinate[label={180:\text{(0,0,0)}}] (OVL) at (0,12,12);
%	\coordinate[label={-120:\text{(0,1,0)}}] (OVR) at (12,12,12);
%	\coordinate[label={-86:\text{(1,0,1)}}] (UHL) at (0,0,0);
%	\coordinate[label={0:\text{(1,1,1)}}] (UHR) at (12,0,0);
%	\coordinate[label={180:\text{(0,0,1)}}] (OHL) at (0,12,0);
%	\coordinate[label={0:\text{(0,1,1)}}] (OHR) at (12,12,0);
%	
%	\draw[fill=black] (UVL) circle (8pt);
%	\draw[fill=black] (UVR) circle (8pt);
%	\filldraw[color=red] (OVL) circle (8pt);
%	\draw[fill=black] (OVR) circle (8pt);
%	\draw[fill=black] (UHL) circle (8pt);
%	\draw[fill=black] (UHR) circle (8pt);
%	\draw[fill=black] (OHL) circle (8pt);
%	\filldraw[color=black] (OHR) circle (8pt);
%	
%	\draw (UVL) -- node[above] {2} (UVR);
%	\draw (UVL) -- node[above left] {1} (OVL);
%	\draw (UVL) -- node[left] {3} (UHL);
%	\draw (OVR) -- node[above right] {2} (OVL);
%	\draw (OVR) -- node[left] {3} (OHR);
%	\draw (OVR) -- node[above left] {1} (UVR);
%	\draw (OHL) -- node[left] {3} (OVL);
%	\draw (OHL) -- node[above] {2} (OHR);
%	\draw (OHL) -- node[left] {1} (UHL);
%	\draw (UHR) -- node[left] {3} (UVR);
%	\draw (UHR) -- node[left] {1} (OHR);
%	\draw (UHR) -- node[above left] {2} (UHL);	
%	
%	\draw[line width=1.5pt,->] (22,8,12) -- node[above] {1. Nacht} (26,8,12); 	
%	\end{tikzpicture}
%	\qquad
%	\hspace*{-0.7cm}
%	\begin{tikzpicture}[thick,scale=0.2]
%	\coordinate[label={180:\text{(1,0,0)}}] (UVL) at (0,0,12);
%	\coordinate[label={0:\text{(1,1,0)}}] (UVR) at (12,0,12);
%	\coordinate[label={180:\text{(0,0,0)}}] (OVL) at (0,12,12);
%	\coordinate[label={-120:\text{(0,1,0)}}] (OVR) at (12,12,12);
%	\coordinate[label={-86:\text{(1,0,1)}}] (UHL) at (0,0,0);
%	\coordinate[label={0:\text{(1,1,1)}}] (UHR) at (12,0,0);
%	\coordinate[label={180:\text{(0,0,1)}}] (OHL) at (0,12,0);
%	\coordinate[label={0:\text{(0,1,1)}}] (OHR) at (12,12,0);
%	
%	\draw[fill=black] (UVL) circle (8pt);
%	\draw[fill=black] (UVR) circle (8pt);
%	\filldraw[color=red] (OVL) circle (8pt);
%	\draw[fill=black] (OVR) circle (8pt);
%	\draw[fill=black] (UHL) circle (8pt);
%	\draw[fill=black] (UHR) circle (8pt);
%	\draw[fill=black] (OHL) circle (8pt);
%	\filldraw[color=black] (OHR) circle (8pt);
%	
%	\draw (UVL) -- node[above left] {1} (OVL);
%	\draw (OVR) -- node[above right] {2} (OVL);
%	\draw (OHL) -- node[left] {3} (OVL);		
%	\end{tikzpicture}
%	\caption{Übergang vom ersten zum zweiten Tag im \textit{cheating husbands}-Rätsels mit zusätzlicher Kommunikation}
%	\label{kom_(0,0,0)}
%\end{figure}
%
%Anders als in \cite{moses1986cheating} behauptet wird, kann das Protokoll nicht so erweitert werden, dass es alle Probleminstanzen in drei Tagen löst und für den Fall, dass es keinen untreuen Ehemann gibt, korrekt funktioniert.(\textbf{Stimmt das?})
%Wie wir bereits gezeigt haben, braucht man im Allgemeinen zwei Nächte, um festzustellen, von welchen beiden Anzahlen untreuer Ehemänner die Frauen im System wissen. Normalerweise ist das Problem dann auch gelöst, die Frauen, die von der kleineren Anzahl (der beiden verbliebenen Möglichkeiten) untreuer Ehemänner ausgehen, erschießen ihre Ehemänner. Dies ist richtig, denn entweder wissen alle Ehefrauen von gleich vielen untreuen Ehemännern (alle waren untreu) oder die mit der kleineren Anzahl kennen einen zu wenig, weil sie selbst betrogen wurden. 
%Dieses Vorgehen ist allerdings nicht mehr korrekt, wenn es tatsächlich keinen einzigen untreuen Ehemann gibt, denn in diesem Fall wissen zwar alle Ehefrauen von gleich vielen untreuen Ehemännern, aber niemand ist betrogen worden.
%Es wäre dementsprechend mindestens noch eine weitere Nacht notwendig, um zu beispielsweise zu unterscheiden, ob tatsächlich alle Frauen von der gleichen Anzahl untreuer Ehemänner wissen, oder es welche gibt, die mehr als die anderen kennen. (\textbf{kann ich das beweisen?})


\section{Fazit}
\label{Zusammenfassung}
In dieser Seminar-Arbeit wurde das Thema von Wissen in verteilten Systemen behandelt.
In Kapitel 1 wurde nach einen kurzen Einführung zunächst das \textit{cheating husbands}-Rätsel vorgestellt und eine korrekte Lösung dazu präsentiert und bewiesen, danach wurden noch einige Schwerpunkte des Themas aufgezählt.
Anschließend wurde in Kapitel 2 das Konzept von möglichen Welten eingeführt. Mit Hilfe von Kripkte-Modellen, sowie den möglichen Welten, konnte dann die Semantik der drei Logikoperatoren des Wissens eingeführt werden: Der Wissensoperator, der \textit{jeder weiß}-Operator und der \textit{gemeinsames Wissen}-Operator.
Das Kripke-Modelle zumindest für kleine Systeme intuitiv visuell dargestellt werden können, wurde dann im Kapitel 3 behandelt. Dabei wurden neue Definitionen für den \textit{jeder weiß}-Operator und den \textit{gemeinsames Wissen}-Operator eingeführt, da diese einfacher an den Graphen zu den Kripke-Modellen nachvollzogen werden können. Die Übereinstimmung der formalen und visuellen Definitionen konnte anhand einiger Beispiele und Überlegungen nachvollzogen werden.\\
Darauffolgend in Kapitel 4 wurde dann die Methodik von asynchronen, sowie synchronen Systemen dargestellt. Es wurde auf die Probleme dieser Systeme hingewiesen, sowie der Unterschied zwischen schwachen und starken synchronen Systemen erläutert.\\
Daran anschließend wurden in Kapitel 5 Erweiterungen für asynchrone Systeme dargestellt. Des Weiteren wurde der Three-Phase-Algorithmus behandelt, um einen praktischen Ansatz dieser Thematik aufzuzeigen. \\
Im Kapitel 6 wurde dann noch einmal vom Nachrichtensystem abstrahiert um zu analysieren wie sich zusätzliche Kommunikationsmöglichkeiten auswirken.
Dabei wurde gezeigt, dass das Problem in konstanter Zeit gelöst werden kann, wenn nur ein wenig zusätzliche Kommunikation erlaubt wird. Gegenüber der Lösung zum Problem ohne die zusätzliche Kommunikation, die lineare Zeit beanspruchte, ist dies eine signifikante Verbesserung.\medskip

Nicht in unserer Arbeit behandelt haben wir die Fragen, wie man Wissenstransfer mittels Nachrichtenketten formalisieren kann und wie sich die Größe der Matrizen logischer Uhren auf das Wissen in asynchronen Nachrichtensystemen auswirkt.\medskip

Das Thema von verteilten Wissen spielt immer dann eine große Rolle, wenn es aufwendig ist neues Wissen zu erlangen, beispielsweise durch Nachrichten, und daher das vorhandene lokale Wissen eines Agenten möglichst vollständig ausgenutzt werden soll.
Solche Situationen treten beispielsweise für Systeme mit mehreren Robotern auf, die über geringe Kommunikationsmöglichkeiten verfügen und dennoch gemeinsam ein Problem lösen sollen.


\bibliographystyle{plain}
\bibliography{literatur}

\end{document}
