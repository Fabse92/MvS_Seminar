\section{Veranschaulichung mit Kripke-Modellen}
\label{Kripke-Modelle}
Das Kripke-Modell eines verteilten Systems mit kleiner Anzahl Prozesse kann intuitiv als Graph dargestellt werden (vgl. \cite{kshemkalyani2011distributed} S. 285 ff.). Dabei entspricht jeder Systemzustand $s\in S$ einem Knoten, der mit der Belegung der atomaren Aussagen $\Phi$ beschriftet wird. Eine Kante wird zwischen zwei Knoten gezeichnet, wenn es mindestens einen Prozess gibt, der die repräsentierten Systemzustände nicht voneinander unterscheiden kann. Sie wird mit den Nummern aller Prozesse beschriftet für die dies gilt.
Der Graph ist dementsprechend bidirektional und reflexiv.\\
\textbf{Beispiel}\medskip

Mit Hilfe der Erreichbarkeit in der Kripke-Struktur kann definiert werden, wie Wissen im Graphen repräsentiert ist.


\subsection{Kripke-Modelle zum \textit{cheating husbands}-Rätsel}
