\section{Fazit}
\label{Zusammenfassung}
In dieser Seminar-Arbeit wurde das Thema von Wissen in verteilten Systemen behandelt.
In Kapitel 1 wurde nach einen kurzen Einführung zunächst das \textit{cheating husbands}-Rätsel vorgestellt und eine korrekte Lösung dazu präsentiert und bewiesen, danach wurden noch einige Schwerpunkte des Themas aufgezählt.
Anschließend wurde in Kapitel 2 das Konzept von möglichen Welten eingeführt. Mit Hilfe von Kripkte-Modellen, sowie den möglichen Welten, konnte dann die Semantik der drei Logikoperatoren des Wissens eingeführt werden: Der Wissensoperator, der \textit{jeder weiß}-Operator und der \textit{gemeinsames Wissen}-Operator.
Das Kripke-Modelle zumindest für kleine Systeme intuitiv visuell dargestellt werden können, wurde dann im Kapitel 3 behandelt. Dabei wurden neue Definitionen für den \textit{jeder weiß}-Operator und den \textit{gemeinsames Wissen}-Operator eingeführt, da diese einfacher an den Graphen zu den Kripke-Modellen nachvollzogen werden können. Die Übereinstimmung der formalen und visuellen Definitionen konnte anhand einiger Beispiele und Überlegungen nachvollzogen werden.\\
Darauffolgend in Kapitel 4 wurde dann die Methodik von asynchronen, sowie synchronen Systemen dargestellt. Es wurde auf die Probleme dieser Systeme hingewiesen, sowie der Unterschied zwischen schwachen und starken synchronen Systemen erläutert.\\
Daran anschließend wurden in Kapitel 5 Erweiterungen für asynchrone Systeme dargestellt. Des Weiteren wurde der Three-Phase-Algorithmus behandelt, um einen praktischen Ansatz dieser Thematik aufzuzeigen. \\
Im Kapitel 6 wurde dann noch einmal vom Nachrichtensystem abstrahiert um zu analysieren wie sich zusätzliche Kommunikationsmöglichkeiten auswirken.
Dabei wurde gezeigt, dass das Problem in konstanter Zeit gelöst werden kann, wenn nur ein wenig zusätzliche Kommunikation erlaubt wird. Gegenüber der Lösung zum Problem ohne die zusätzliche Kommunikation, die lineare Zeit beanspruchte, ist dies eine signifikante Verbesserung.\medskip

Nicht in unserer Arbeit behandelt haben wir die Fragen, wie man Wissenstransfer mittels Nachrichtenketten formalisieren kann und wie sich die Größe der Matrizen logischer Uhren auf das Wissen in asynchronen Nachrichtensystemen auswirkt.\medskip

Das Thema von verteilten Wissen spielt immer dann eine große Rolle, wenn es aufwendig ist neues Wissen zu erlangen, beispielsweise durch Nachrichten, und daher das vorhandene lokale Wissen eines Agenten möglichst vollständig ausgenutzt werden soll.
Solche Situationen treten beispielsweise für Systeme mit mehreren Robotern auf, die über geringe Kommunikationsmöglichkeiten verfügen und dennoch gemeinsam ein Problem lösen sollen.