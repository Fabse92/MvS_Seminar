\section{Logik des Wissens}
\label{Logik}
Damit möglichst intuitiv Logik über das Wissen in verteilten Systemen betrieben werden kann, werden Logikoperatoren aus der epistemischen Logik genutzt, mit denen eine Logik (bspw. Aussagenlogik) erweitert werden kann.

Fagin et al. führen dazu zunächst das Konzept von \textbf{möglichen Welten} ein (vgl. \cite{fagin2003reasoning} S.15 ff.). 
Da ein einzelner Prozess in einem verteilten System nur eingeschränkte Informationen über die Zustände in den anderen Prozessen hat, kann es verschiedene Systemzustände geben, die er anhand seiner lokalen Informationen bzw. Zustände nicht voneinander unterscheiden kann. Die möglichen Welten $P_i$ werden daher definiert als: Die Vereinigung aller Systemzustände, die mit den vorhanden Informationen eines Prozesses $i$ vereinbar sind.
Mit Hilfe der möglichen Welten können Aussagen über eine Formel $\phi$ gemacht werden:
Ein Prozess $i$ \textit{weiß}, dass $\phi$ gilt ($K_i\phi$), wenn in allen Welten, die er für möglich hält, $\phi$ gilt.
%Die Syntax zur Wissenslogik kann entsprechend ausgedrückt werden als:
%\begin{align*}
%\phi &::= p | \neg\phi_1 | \phi_1 \land \phi_2 | K_i\phi
%\end{align*}
%Wobei $p\in P$ eine atomare Aussagenlogische Formel ist, $\phi_1$ und $\phi_2$ sind Formeln unserer Wissenslogik und $i \in \{1,...,n\}$ in einem verteilten System mit $n$ Prozessen.
Die Semantik zu den Logikoperatoren des Wissens kann mit Hilfe von \textbf{Kripke-Modellen} definiert werden (vgl. \cite{kshemkalyani2011distributed} S. 285 f.).
Ein Kripke-Modell $M$ zu einem verteilten System mit $n$ Prozessen und einer Menge atomarer Aussagen $\Phi$ ist ein Tupel $(S,\pi,\mathcal{K}_1,...,\mathcal{K}_n)$ mit der Menge möglicher Systemzustände $S$, der Belegungsfunktion $\pi$, die jeder atomaren Aussage in jedem Zustand $s$ einen Wahrheitswert zuordnet ($\forall s\in S, \pi(s):\Phi \rightarrow \{0,1\}$) und einer binären Relation $\mathcal{K}_i$ auf $S$ für jeden Prozess $i$, die jedem Prozess $i$ seine möglichen Welten, also die Menge von Systemzuständen, die er nicht vom tatsächlichen Systemzustand unterscheiden kann, zuordnet. $\mathcal{K}_i$ enthält für alle Zustände $s \in S$ alle Tupel ($s,t$) von Zuständen, die anhand seiner lokalen Informationen nicht unterscheidbar sind. \medskip

Damit ergeben sich folgende Definitionen der Wissensoperatoren:
\begin{itemize}
\item (M,s) $\vDash \phi$, falls $\pi(s)(\phi) = true$
\item (M,s) $\vDash K_i(\phi)$, falls für alle t $\in S$ mit $(s,t) \in \mathcal{K}_i $ gilt: (M,t) $\vDash \phi$
\item (M,s) $\vDash E^1(\phi)$, falls (M,s) $\vDash\land_{i\in N}K_i(\phi)$
\item (M,s) $\vDash E^{k+1}(\phi)$ mit $k\ge 1$, falls (M,s) $\vDash\land_{i\in N}K_i(E^k(\phi))$
\item (M,s) $\vDash C(\phi)$, falls (M,s) $\vDash\land_{k\in Z^*}E^k(\phi)$
\end{itemize}

Dabei ist K der bereits erwähnte Wissensoperator, E der \textit{jeder weiß}-Operator und C der \textit{gemeinsames Wissen}-Operator.
Der \textit{jeder weiß}-Operator wird rekursiv für höhere Ebenen definiert: $E^1(\phi)$ bedeutet, dass jeder Prozess im System weiß, dass $\phi$ gilt; $E^2(\phi)$ bedeutet, dass jeder Prozess weiß, dass jeder Prozess weiß, dass $\phi$ gilt, und so weiter.
Der \textit{gemeinsames Wissen}-Operator kann intuitiv so verstanden werden, dass es definitiv für keinen Prozess einen Zweifel daran gibt, dass den anderen Prozessen $\phi$ bekannt ist. Dies gilt beispielsweise in unserer ersten Version des \textit{cheating husbands}-Rätsels für die Information, dass es mindestens einen untreuen Ehemann gibt, nachdem die Königin dies in der Gegenwart aller Ehefrauen verkündet hat. 