\documentclass[t]{beamer} %% alles am TOP ausrichten, c=CENTER ist default.
%% \documentclass[pausesections]{beamer}
% \documentclass[handout]{beamer}

% \usepackage{pgfpages}
%% \pgfpagesuselayout{4 on 1}[a4paper, border shrink 5mm]
% \pgfpagesuselayout{2 on 1}[a4paper, border shrink 5mm]
%% \pgfpagesuselayout{4 on 1}[a4paper, landscape]

% \usepackage[german]{babel}
\usepackage[latin1]{inputenc}
% \usepackage{times}
\usepackage[T1]{fontenc}
\usepackage{graphicx}
\usepackage{amscd}


\setlength{\parskip}{2mm}

% % don't show title and author in the footer
% \beamertemplatefootempty
% % Navigationssymbole ausblenden:
% \setbeamertemplate{navigation symbols}{}
% % \setbeamertemplate{footline}{}
% %% ohne Navigation:
% \beamertemplatenavigationsymbolsempty  

% don't show structure in header
% \beamertemplateheadempty
% fix block title font for acroread
% \setbeamerfont*{block title}{}

\mode<presentation>
{
  % \usetheme{Madrid} %ohne navigation
  % \usetheme{Warsaw}  
  % \usetheme{Hannover} 
  \usetheme{PaloAlto} 
  \setbeamercovered{transparent}
}


\title{  Verteiltes Wissen } 
%\subtitle{Reasoning with knowledge}
\author[Schubert, Schmid]{Danny Schubert, Jan Fabian Schmid}
\date[]{\today} 
\institute[UHH] {Seminar im Modul MvS\\
	Fachbereich Informatik \\ Universit�t Hamburg }

% Dies wird lediglich in den PDF Informationskatalog einf�gt. 
\subject{Slides}

\pgfdeclareimage[height=1.5cm]{university-logo}{UHH-Logo-cut}
\logo{\pgfuseimage{university-logo}}


% Folgendes sollte gel�scht werden, wenn man nicht am Anfang jedes
% Unterabschnitts die Gliederung nochmal sehen m�chte.
\AtBeginSection[]
{
  \begin{frame}<beamer>
    \frametitle{Outline}
    \small\tableofcontents[currentsection,currentsubsection]
  \end{frame}
}


% Falls Aufz�hlungen immer schrittweise gezeigt werden sollen, kann
% folgendes Kommando benutzt werden:
% \beamerdefaultoverlayspecification{<+->}



\begin{document}

%%%%%%%%%%%%%%%%%%%%%%%%%%%%%%%%%%%%%%%%%%%%%%%%%%%%%% 
\begin{frame}
  \titlepage
\end{frame}

\section{Einf�hrung}

\begin{frame}
  % \small
  % \footnotesize
  \frametitle{ Betrachtung von Wissen in verteilten Systemen }

  \begin{itemize}
  	\item Prozesse im verteilten System verf�gen nur �ber einen Teil der Systeminformationen
  	\item Kommunikation in verteilten Systemen ist aufwendig 
  	% hoher Berechnungsaufwand, oder gar nicht m�glich
  	% Beispielsweise mehrere Roboter, die eine Aufgabe gemeinsam bearbeiten  	
  	\item Prozesse treffen anhand unvollst�ndiger Informationen Entscheidungen
  	\item \textbf{Ziel:} Wissen �ber den Systemzustand ausgehend von lokalen Zust�nden ableiten
  \end{itemize}

\end{frame}

\begin{frame}
  \frametitle{ Fragestellungen }

  Folgende Fragen und Probleme ergeben sich:

  \begin{itemize}
  	\item Wie k�nnen wir Logik mit Wissen betreiben?
  	\item K�nnen wir Wissen im System intuitiv darstellen?
  	\item Welchen Einfluss hat das Kommunikationssystem auf das Wissen?
  	\item Wie gehen wir mit dem Konsens-Problem um?
  \end{itemize}

\end{frame}

\begin{frame}
	\frametitle{ Das \textit{cheating husbands}-R�tsel }
	Induktionsr�tsel zur Veranschaulichung von verteiltem Wissen
	\begin{itemize}
		\item Jede Ehefrau wei�, welche der anderen Ehefrauen betrogen wurden
		\item Keine Ehefrau wei�, ob sie selbst betrogen wurde
		\item Es gibt mindestens einen untreuen Ehemann
		\item Die Frauen d�rfen nicht untereinander kommunizieren
		\item Findet eine Ehefrau heraus, dass sie betrogen wurde, erschie�t sie ihren Ehemann um Mitternacht
	\end{itemize}
	
\end{frame}

\begin{frame}
	\frametitle{ L�sung des \textit{cheating husbands}-R�tsels }
	
	\begin{block}{Behauptung}
		Wenn n Ehem�nner untreu waren, so werden diese alle in der n-ten Nacht erschossen.
	\end{block}
	
	Beweis mittels vollst�ndiger Induktion �ber die Anzahl untreuer Ehem�nner $n$:
	
	\begin{itemize}
		\item \textbf{Induktionsanfang $(n=1)$:} Die betrogene Ehefrau wei� von keinem untreuen Ehemann. Da es mindestens einen untreuen Ehemann gibt, ist sie betrogen worden. Die betrogene Frau erschie�t ihren Ehemann in der ersten Nacht. 
	\end{itemize}
	
\end{frame}

\section{Logik des Wissens}

\begin{frame}
	\frametitle{ Logikoperatoren des Wissens }
	
	\begin{itemize}
		\item ToDo
	\end{itemize}
	
\end{frame}


\section{Kripke-Modelle}

\begin{frame}
	\frametitle{ Logikoperatoren in Kripke-Modellen }
	
	\begin{itemize}
		\item ToDo
	\end{itemize}
	
\end{frame}

\begin{frame}
	\frametitle{ Veranschaulichung mit Kripke-Modellen }
	
	\begin{itemize}
		\item ToDo
	\end{itemize}
	
\end{frame}

\begin{frame}
	\frametitle{ Das \textit{cheating husbands}-R�tsel veranschaulicht  }
	
	\begin{itemize}
		\item ToDo
	\end{itemize}
	
\end{frame}

\section{Wissen in synchronen und asynchronen Systemen}

\begin{frame}
	\frametitle{ Titel}
	
	\begin{itemize}
		\item ToDo
	\end{itemize}
	
\end{frame}

\section{Varianten des gemeinsamen Wissens}

\begin{frame}
	\frametitle{ Titel }
	
	\begin{itemize}
		\item ToDo
	\end{itemize}
	
\end{frame}

\section{Zus�tzliche Kommunikation}

\begin{frame}
	\frametitle{ Das \textit{cheating husbands}-R�tsel mit zus. Kommunikation }
	
	\begin{itemize}
		\item ToDo
	\end{itemize}
	
\end{frame}

\section{Fazit}

\begin{frame}
  \frametitle{Zusammenfassung }
	
	\begin{itemize}
		\item ToDo
	\end{itemize}
	
\end{frame}

\end{document}
